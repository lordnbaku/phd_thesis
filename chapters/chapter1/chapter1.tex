%=============================================================
%  CHAPTER 1 – INTRODUCTION
%  (LaTeX source for a PhD thesis)
%=============================================================
\documentclass[12pt,a4paper]{report}
\usepackage[utf8]{inputenc}
\usepackage[T1]{fontenc}
\usepackage{lmodern}
\usepackage{amsmath,amssymb,amsthm}
\usepackage{graphicx}
\usepackage{booktabs}
\usepackage{caption}
\usepackage{subcaption}
\usepackage{setspace}
\onehalfspacing
\usepackage[natbib=true,style=apa,backend=biber]{biblatex}
\addbibresource{references.bib}   % <-- put all your .bib entries here

% ---- page layout ------------------------------------------------
\usepackage[left=3.5cm,right=3.0cm,top=3.0cm,bottom=3.0cm]{geometry}
\usepackage{fancyhdr}
\pagestyle{fancy}
\fancyhf{}
\rhead{\thepage}
\lhead{\nouppercase{\leftmark}}

% ---- section numbering depth ------------------------------------
\setcounter{secnumdepth}{3}
\setcounter{tocdepth}{2}

% ---- custom commands --------------------------------------------
\newcommand{\RQ}[1]{\textbf{RQ#1}}
\newcommand{\ie}{i.e.,}
\newcommand{\eg}{e.g.,}
\newcommand{\etal}{\textit{et al.}}

\begin{document}

\chapter{Introduction}
\label{ch:intro}

\section{Background and Context}
Nigeria’s banking sector has repeatedly faced severe crises triggered by **regulatory ambiguity**, **inconsistent enforcement**, and **elevated non-performing loan (NPL) ratios**. The 2009 meltdown saw NPLs exceed 30\%, necessitating a ₦620 billion intervention \citep{sanusi2014}. Post-crisis reforms, notably the Central Bank of Nigeria’s (CBN) 2014 Code of Corporate Governance, aimed to harmonise directives on risk management, compliance, and loan provisioning. Nevertheless, overlapping mandates from the Securities and Exchange Commission (SEC) and the National Deposit Insurance Corporation (NDIC) continue to create interpretive challenges and erode compliance efficacy \citep{agbloyor2013,osemeke2016}.

Recent data underline the fragility of these reforms. By mid-2024 the aggregate NPL ratio stood at 4.2\%, rising to 4.6\% by year-end amid naira devaluation and inflationary pressures \citep{cbn2024}. The scheduled lifting of CBN forbearance measures in June 2025 is projected to push NPLs a further 1–2 percentage points \citep{spglobal2025}. These trends highlight the need for a deeper understanding of **corporate-governance mechanisms**, **audit-committee independence**, and **regulatory clarity** as levers to curb credit risk.

\section{Research Gap}
Empirical work on the governance–NPL nexus in emerging markets remains fragmented. While developed-market studies link transparent regulation to lower NPLs \citep{erkens2020}, African evidence is inconclusive, often attributing high NPLs to enforcement gaps rather than regulatory multiplicity per se \citep{asare2021}. In Nigeria, prior research concentrates on macroeconomic drivers or board-independence proxies but neglects **mediation pathways** through compliance practices \citep{ofoeda2022,atoi2021}. No study to date has simultaneously examined:
\begin{itemize}
  \item the multi-dimensional impact of **corporate-governance indices** (disclosure, practice, compliance);
  \item the **mediating role of financial-reporting integrity** in the audit-committee–NPL relationship;
  \item **temporal shifts** in disclosure credibility post-CBN reforms;
  \item the **causal effect of perceived regulatory clarity** amid ongoing CBN/SEC harmonisation.
\end{itemize}

\section{Aim and Research Questions}
This thesis seeks to **quantify and contextualise the influence of corporate governance and regulatory clarity on non-performing loan ratios in Nigerian deposit-money banks (DMBs)** over the period 2009--2025. Four interrelated research questions guide the investigation:

\begin{enumerate}
  \item[\RQ{1}] To what extent do **corporate-governance indices** (disclosure, practice, and compliance) influence non-performing loan ratios in Nigerian DMBs over the period 2009--2024, controlling for bank size, capital adequacy, and macroeconomic factors? \hfill \textit{(Quantitative panel analysis)}
  \item[\RQ{2}] To what extent does **audit-committee independence** reduce non-performing loan ratios in Nigerian DMBs in 2024, and is this effect mediated by strengthened **financial-reporting integrity** as perceived by bank insiders? \hfill \textit{(Mixed-methods mediation)}
  \item[\RQ{3}] The relationship between **corporate-governance disclosure (CGDI)** and **non-performing loan ratios (NPLR)** is statistically insignificant in the pre-reform period (2009--2014) but becomes significantly negative post-reform (2016--2024), due to the exogenous enhancement of disclosure credibility via CBN-mandated enforcement and compliance verification.
  \item[\RQ{4}] Over the period 2020--2025, improvements in **perceived regulatory clarity** (from CBN/SEC harmonisation) will causally reduce NPL ratios in Nigerian DMBs (\textbf{total effect} $\beta < 0$, $p < 0.05$), controlling for macroeconomic shocks.
\end{enumerate}

\section{Significance of the Study}
The study offers **three original contributions**:
\begin{enumerate}
  \item A **multi-dimensional governance index** that disentangles disclosure, practice, and compliance effects on NPLs using 16-year panel data.
  \item The first **mixed-methods mediation test** of audit-committee independence via perceived reporting integrity, blending insider surveys with archival NPL data.
  \item A **quasi-experimental evaluation** of post-reform disclosure credibility (RQ3) and a **prospective causal claim** on regulatory-clarity harmonisation (RQ4) – both policy-relevant as Nigeria moves toward full Basel III adoption.
\end{enumerate}

\section{Theoretical Foundations}
The thesis is anchored in **institutional theory** \citep{dimaggio1983,meyer1977,north1990}, which posits that ambiguous regulatory environments foster **decoupling** – symbolic compliance without substantive risk-management improvements. **Agency theory** complements this by framing audit-committee independence as a safeguard against managerial opportunism that fuels ever-greening of impaired loans. The integration of these lenses allows a nuanced examination of **direct**, **mediated**, and **moderated** pathways linking governance to credit risk.

\section{Structure of the Thesis}
\begin{itemize}
  \item \textbf{Chapter 2} – Literature review, theoretical framework, and hypothesis development.
  \item \textbf{Chapter 3} – Research design, data sources, and econometric specifications.
  \item \textbf{Chapter 4} – Empirical results for RQ1 (panel regressions) and RQ3 (difference-in-differences).
  \item \textbf{Chapter 5} – Mixed-methods analysis for RQ2 (bootstrapped mediation + thematic insights).
  \item \textbf{Chapter 6} – Prospective analysis for RQ4 (instrumental-variable / propensity-score matching).
  \item \textbf{Chapter 7} – Discussion, policy recommendations, limitations, and avenues for future research.
\end{itemize}

\printbibliography[heading=subbibintoc,title={References for Chapter 1}]

\end{document}
