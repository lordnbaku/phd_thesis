% University of Buckingham — PhD Thesis: Chapter 1 (Ready-to-compile)
% Filename: UB_PhD_Chapter1.tex
% Defaults used: A4 paper, 12pt, 1.5 line spacing, left margin 1.5 in, Times (newtx), biblatex-apa style using biber.
% This single .tex file contains a minimal preamble, an embedded references.bib (via filecontents), and Chapter 1 content.
% Compile instructions (from terminal on Arch Linux):
% 1. sudo pacman -S texlive-most biber (if not installed)
% 2. pdflatex UB_PhD_Chapter1.tex
% 3. biber UB_PhD_Chapter1
% 4. pdflatex UB_PhD_Chapter1.tex
% 5. pdflatex UB_PhD_Chapter1.tex

\RequirePackage{filecontents}
\begin{filecontents}{references.bib}
@book{jensen1976,theauthor = {Jensen, M. C. and Meckling, W. H.}, title = {Theory of the firm: Managerial behavior, agency costs, and ownership structure}, year = {1976}}
@article{sanusi2014, author = {Sanusi, L. S.}, title = {Banking crisis in Nigeria: causes and responses}, year = {2014}}
@misc{cbn2024, author = {Central Bank of Nigeria}, title = {Statistical Bulletin 2024}, year = {2024}}
% --- Add your full .bib entries below ---
\end{filecontents}

\documentclass[12pt,a4paper]{report}
\usepackage[a4paper,inner=1.5in,outer=1in,top=1in,bottom=1in]{geometry}
\usepackage{newtxtext,newtxmath}
\usepackage{setspace}
\onehalfspacing
\usepackage{microtype}
\usepackage[utf8]{inputenc}
\usepackage[T1]{fontenc}
\usepackage{hyperref}
\hypersetup{colorlinks=true,linkcolor=black,urlcolor=blue}
\usepackage{csquotes}
\usepackage[style=apa,sortcites=true,backend=biber]{biblatex}
\addbibresource{references.bib}
\usepackage{tocloft}
\setcounter{secnumdepth}{4}
\setcounter{tocdepth}{3}
\usepackage{titlesec}
\titleformat{\chapter}[hang]{\normalfont\LARGE\bfseries}{\thechapter.}{1em}{}
\titleformat{\section}[hang]{\normalfont\Large\bfseries}{\thesection}{1em}{}
\titleformat{\subsection}[hang]{\normalfont\large\bfseries}{\thesubsection}{1em}{}

% Title page metadata: edit as required
\newcommand{\ThesisTitle}{Corporate Governance and Non-performing Loans: Chapter One}
\newcommand{\AuthorName}{OluwaseunIsaiah Jimoh-Ibrahim}
\newcommand{\DegreeName}{Doctor of Philosophy}
\newcommand{\DeptName}{Department of Business and Management}
\newcommand{\UniversityName}{University of Buckingham}
\newcommand{\SupervisorName}{Supervisor: Dr. Singh}
\newcommand{\SubmissionDate}{\today}

\begin{document}

% Title page (simple template; adjust to UB official requirements if you have a template)
\begin{titlepage}
  \centering
  \vspace*{2cm}
  {\Huge \bfseries \ThesisTitle \\}
  \vspace{1.5cm}
  {\Large \AuthorName \\}
  \vfill
  {\Large \DegreeName \\}
  {\Large \DeptName \\}
  {\Large \UniversityName \\}
  \vspace{1cm}
  {\Large \SupervisorName \\}
  \vspace{1.5cm}
  {\Large Submitted: \SubmissionDate \\}
\end{titlepage}

\pagenumbering{roman}
\tableofcontents
\cleardoublepage
\listoffigures
\cleardoublepage
\listoftables
\cleardoublepage
\pagenumbering{arabic}

% --- Chapter 1 content starts here (converted from your Word doc) ---
\chapter{Introduction}
\section{1.1Background to the Study}
The Nigerian deposit money banking (DMB) sector has long been a cornerstone of the nation's economic infrastructure, facilitating credit allocation, financial intermediation, and systemic stability amid volatile macroeconomic conditions. However, its evolution has been punctuated by recurrent governance crises that have precipitated institutional failures and amplified default risks, as manifested in elevated non-performing loan (NPL) ratios. Historically, the sector's governance challenges trace back to the pre-independence era, where colonial banking structures emphasized extractive practices over indigenous accountability mechanisms, setting a precedent for opacity and elite capture (Falola & Heaton, 2008). Post-independence, the indigenization policies of the 1970s aimed to localize control but inadvertently entrenched familial and political influences in board compositions, fostering agency conflicts that undermined prudent risk management (Adegbite et al., 2012). This institutional legacy culminated in the distress era of the late 1980s and early 1990s, during which over 425 bank failures were recorded, driven by insider lending, inadequate disclosure, and regulatory laxity, with NPL ratios surging beyond 37% in some instances (NDIC, 2010; Sanusi, 2014; Udo, 2020). Such episodes not only eroded depositor confidence but also highlighted the interdependencies between weak corporate governance apparatuses and systemic vulnerabilities, necessitating a deeper empirical scrutiny of mechanisms that could attenuate these risks, as framed by the research questions guiding this thesis.

Building upon this foundation, the 2004–2005 banking consolidation under the Central Bank of Nigeria (CBN) represented a pivotal institutional response, mandating a minimum capital base of ₦25 billion and reducing the number of banks from 89 to 25 through mergers and acquisitions (Soludo, 2004). This reform sought to enhance resilience by promoting economies of scale and professionalizing governance structures, yet it inadvertently amplified concentration risks and perpetuated cronyism in larger entities. Correspondingly, the 2009 financial crisis exposed persistent deficiencies, as NPL ratios exceeded 37% amid revelations of fraudulent practices, such as concealed toxic assets and executive malfeasance, leading to the bailout of eight banks and the dismissal of several CEOs (Sanusi, 2010). These events underscored the limitations of prior reforms in addressing core governance erosions, including board tenure excesses and insufficient independent oversight, which institutional theory posits as artifacts of fragmented regulatory environments that fail to align incentives with stakeholder interests (North, 1990; Adegbite, 2015). The crisis precipitated a wave of post-2009 interventions, including the establishment of the Asset Management Corporation of Nigeria (AMCON) to absorb non-performing assets, yet NPL ratios remained stubbornly high, averaging 14.8% between 2010 and 2015, reflecting the enduring impact of governance lapses on credit risk (CBN, 2016).

In response to these entrenched challenges, successive reforms have aimed to institutionalize robust governance frameworks, drawing on global standards while adapting to Nigeria's emerging market context. The CBN Code of Corporate Governance for Banks 2014 introduced mandates for board diversity, with at least 50% independent directors, and limits on CEO tenure to two five-year terms, emphasizing disclosure and compliance to mitigate agency problems (CBN, 2014). This was complemented by the Nigerian Code of Corporate Governance (NCCG) 2018, issued by the Financial Reporting Council of Nigeria (FRCN), which advocated for stewardship principles through enhanced audit committee independence and risk management protocols, aligning with signalling theory by positing that credible disclosures signal managerial integrity to investors (Spence, 1973; FRCN, 2018). Further, the Companies and Allied Matters Act (CAMA) 2020 reinforced these efforts by mandating executive vetting and prohibiting interlocking directorships, while the CBN Guidelines on Corporate Governance 2023 harmonized requirements with Basel III pillars, focusing on capital adequacy and macroeconomic stress testing (CAMA, 2020; CBN, 2023). Despite these advancements, empirical evidence suggests a disconnect between formal compliance and substantive risk reduction, as NPL ratios hovered around 5–6% in 2023–2024 amid economic shocks like naira devaluation and inflation spikes (CBN, 2024; Agusto & Co., 2024). This mismatch evidences institutional theory's contention that exogenous reforms may activate governance mechanisms unevenly in contexts of regulatory multiplicity, where overlapping mandates from the CBN, Securities and Exchange Commission (SEC), and Nigeria Deposit Insurance Corporation (NDIC) engender enforcement inconsistencies (DiMaggio & Powell, 1983; Olojede et al., 2020).

At its core, corporate governance in banking encompasses mechanisms—such as board oversight, disclosure protocols, and compliance frameworks—that ensure transparency, accountability, and prudent decision-making to safeguard stakeholder interests (OECD, 2023). In the Nigerian DMB sector, these mechanisms are theoretically anchored in agency theory, which highlights the need to align managerial actions with shareholder value through monitoring, and stewardship theory, which emphasizes intrinsic motivations for ethical conduct (Jensen & Meckling, 1976; Davis et al., 1997). Yet, poor governance manifests in elevated NPL ratios, defined as the proportion of loans overdue by 90 days or more, serving as a proxy for default risk and credit quality deterioration (IMF, 2023). The conceptual nexus is evident: deficient disclosure obscures risk exposures, weak practices enable insider abuses, and non-compliance erodes regulatory trust, collectively precipitating NPL escalations that strain capital buffers and threaten solvency (Abdulmalik & Ahmad, 2020). For instance, pre-reform periods (2009–2014) saw insignificant governance-NPL linkages due to lax enforcement, while post-reform shifts (2016–2024) suggest negative associations attributable to enhanced credibility, though mediated by macroeconomic confounders like oil price volatility and GDP fluctuations (Natufe & Evbayiro-Osagie, 2023).

Comparatively, global benchmarks under Basel III and OECD principles advocate for integrated governance that embeds risk management into board responsibilities, achieving NPL ratios below 2% in mature economies like the United Kingdom and South Africa through stringent independent audits and harmonized regulations (BCBS, 2015; OECD, 2023). In contrast, Nigeria's realities reveal a persistent gap: despite reform convergence, institutional isomorphism—where banks mimic global standards superficially without internalizing them—has yielded suboptimal outcomes, as evidenced by comparative studies in sub-Saharan Africa showing Ghana's more unified regulatory approach correlating with lower NPLs (Agyemang & Appiah, 2017; Adegbite et al., 2020). This disparity underscores the need for context-specific inquiries that disentangle governance indices' modulating effects on NPLs, incorporating perceptual mediations and reform-induced shifts.

These historical, reformative, and conceptual dimensions collectively illuminate a critical mismatch: while formal governance apparatuses have evolved, their translation into reduced default risks remains uneven amid regulatory fragmentation and economic volatilities. To elucidate these interdependencies, the following subsections delve into specific facets of governance challenges, default risks, regulatory multiplicity, and broader institutional hurdles in the Nigerian DMB sector, setting the stage for the research questions that guide this thesis.

\subsection{1.1.1 Corporate Governance Issues: Emphasis on Transparency and Accountability}
Corporate governance in Nigeria’s deposit money banking sector is pivotal for ensuring financial stability, yet it faces systemic challenges that undermine transparency and accountability—two essential pillars for fostering trust and safeguarding stakeholder interests. Transparency ensures stakeholders have access to accurate, timely, and comprehensive information about a bank’s financial and operational performance, while accountability holds management and boards responsible for their actions, aligning their interests with those of shareholders and other stakeholders (Solomon, 2020; Healy & Palepu, 2001). In Nigeria, these principles are consistently eroded by deficiencies in board oversight, disclosure practices, stakeholder communication, audit controls, and executive compensation practices, leading to reduced investor trust, heightened regulatory scrutiny, and reputational damage, all of which contribute to heightened default risk. This subsection examines these issues within the Nigerian context, drawing on historical legacies, regulatory challenges, and specific cases like Oceanic Bank to highlight their impact, and proposes measures to strengthen governance and mitigate the risk of bank failures.

The board of directors, as the cornerstone of corporate governance, is responsible for overseeing management, setting strategic direction, and ensuring accountability to shareholders (Aguilera & Ruiz Castillo, 2025; OECD, 2023). However, in Nigeria’s banking sector, boards often lack the independence and effectiveness needed to fulfill these roles. Historical government interference, particularly before 1986, and post-independence political appointments have fostered a culture of cronyism, with boards frequently populated by unqualified or politically motivated members (Chinweoda et al., 2020; Nwoke et al., 2022). At Oceanic Bank, for example, the presence of family members such as Mrs. Ibru and Mr. Oboden Ibru on the board enabled insider loans, with over 75% of non-performing loans in 2009 linked to Ibru family-controlled companies, reflecting a severe lack of independent oversight (Sanusi, 2009). Compounding this issue, 61% of Nigerian banks lack independent nomination committees, allowing concentrated ownership to dominate appointments and prioritize management’s self-interest over stakeholders, as seen in the collapses of Intercontinental Bank and Bank PHB (Ogbechie, 2016; Owolabi, 2013). This lack of independence undermines the board’s ability to hold management accountable, leading to financial mismanagement and increased default risk.

Transparent disclosure and reporting practices are vital for stakeholders to assess a bank’s financial health, yet Nigerian banks frequently engage in financial misstatements and fail to provide timely information. Many banks deviate from proper accounting standards, exploiting regulatory multiplicity to adopt less stringent codes, which distorts their financial position (Babajide et al., 2020; Osemeke & Adegbite, 2016). During the 2009 banking crisis, for instance, weak oversight allowed banks like Oceanic Bank to conceal non-performing loans through inadequate reporting, delaying stakeholder awareness of financial distress (Olojede et al., 2020). While the shift of Audit Committee reports from voluntary to mandatory has aimed to improve transparency, their effectiveness is often compromised when committees lack independence, functioning at the behest of majority shareholders or management (Healy & Palepu, 2001; Adegbite, 2014). This opacity erodes investor confidence and increases the risk of bank failures, as stakeholders cannot accurately gauge the bank’s stability (Ugochukwu, 2000).

Effective governance also requires robust stakeholder communication and the protection of rights, particularly for minority shareholders, but Nigerian banks often exhibit opaque decision-making and neglect these groups. Management frequently prioritizes its own interests over those of shareholders, a classic agency problem exacerbated by weak board oversight (Osemeke & Adegbite, 2016). In the case of Oceanic Bank, decisions on insider loans and asset transfers to shell companies were made without transparent communication to shareholders, marginalizing minority voices and eroding trust (Sanusi, 2009). Regulatory multiplicity further complicates this issue, as conflicting codes—such as the NCCG’s “Apply and Explain” flexibility versus the CBN’s stricter mandates—create ambiguity in enforcing shareholder rights (Abdulmalik & Ahmad, 2020). The absence of robust whistleblowing mechanisms, a key factor in the 2009 crisis, also stifles stakeholders’ ability to report misconduct, further entrenching opacity in decision-making and undermining accountability (Olojede et al., 2020).

Audit committees and internal controls are critical for ensuring transparency in financial reporting and detecting mismanagement, but in Nigeria, these mechanisms are often weak and subject to external auditor conflicts. Audit committees lack independence, with CAMA Section 359(4)’s mandate for equal representation of directors and shareholders undermined by majority shareholder influence (Adegbite, 2014). During the 2009 crisis, audit committees failed to rigorously review financial statements, allowing fraud and financial impropriety to go undetected (Olojede et al., 2020). External auditor conflicts, similar to those seen in global cases like Enron where auditors faced constraints in investigating company affairs, are also prevalent in Nigeria, weakening the oversight of internal controls and allowing financial misstatements to persist (Mallin, 2010; Rezaee et al., 2003; Hossain & Hou, 2020). These deficiencies increase default risk by obscuring the true financial health of banks.

Executive compensation structures, intended to align management’s interests with those of shareholders, often foster misaligned incentives and lack transparency in Nigeria’s banking sector. Regulatory multiplicity creates discrepancies in remuneration policies, with codes like the SEC Code of 2011 requiring shareholder approval for compensation, while the NCCG allows more flexibility, leading to inconsistent practices (Osemeke & Adegbite, 2016). At Oceanic Bank, undisclosed and excessive compensation to executives tied to the Ibru family facilitated insider dealings, prioritizing personal gain over shareholder interests (Sanusi, 2009). Such non-disclosure obscures accountability, as stakeholders cannot assess whether incentives drive ethical performance, while the lack of performance-linked compensation allows executives to face little consequence for mismanagement, further increasing the risk of financial distress (Al-Shaer et al., 2022; Salehi et al., 2023).

These governance issues have profound consequences for transparency and accountability in Nigeria’s banking sector. Opaque decision-making, financial misstatements, and non-disclosure of compensation reduce investor trust, leaving them unable to make informed decisions, as evidenced by mass deposit withdrawals following the 2009 crisis (Ugochukwu, 2000). Regulatory scrutiny intensifies as agencies like the Central Bank of Nigeria (CBN) and the Financial Reporting Council of Nigeria (FRCN) struggle to enforce compliance amidst conflicting codes, with weak enforcement mechanisms failing to deter governance failures (Nwoke et al., 2023; Adegbite, 2012). Reputational damage follows, as public backlash against governance lapses—similar to global cases like United Airlines in 2017—erodes stakeholder confidence and impacts banks’ market standing (Shepardson, 2017). Collectively, these issues heighten default risk, with 425 bank failures since 1988 underscoring the systemic consequences of governance deficiencies (Udo, 2020).

Addressing these challenges requires targeted measures to strengthen transparency and accountability. Boards must be made more independent by mandating nomination committees to ensure diverse, qualified, and impartial members, reducing the influence of concentrated ownership (Ogbechie, 2016; Fariha et al., 2022). Stricter enforcement of timely and accurate financial reporting, with audit committees empowered to operate independently, is essential to improve disclosure standards (Healy & Palepu, 2001; Al-Shaer et al., 2022). Robust whistleblowing mechanisms and independent external audits can enhance internal controls, detecting and preventing financial impropriety (Olojede et al., 2020; Rezaee et al., 2003). Aligning executive compensation with performance, coupled with mandatory disclosure of remuneration policies, ensures incentives drive ethical behavior and enhance transparency (Salehi et al., 2023). Continuous monitoring through regular governance audits, stakeholder feedback, and regulatory harmonization is vital to sustain these reforms, aligning Nigeria’s banking sector with global best practices (Herbert & Agwor, 2021; Beekes & Brown, 2006).

In conclusion, transparency and accountability are indispensable for effective corporate governance in Nigeria’s deposit money banking sector, yet they are consistently undermined by systemic issues in board oversight, disclosure practices, stakeholder communication, audit controls, and executive compensation. These deficiencies, rooted in historical legacies and exacerbated by regulatory multiplicity, have led to reduced investor trust, heightened regulatory scrutiny, and reputational damage, contributing to 425 bank failures since 1988 and persistent default risk (Udo, 2020; Kafidipe, 2021). Cases like Oceanic Bank illustrate how these governance failures—lacking independent oversight, concealing financial distress, and misaligning incentives—threaten financial stability and stakeholder confidence. Implementing measures such as strengthening board independence, improving disclosure standards, enhancing internal controls, and aligning compensation with performance is critical to restoring trust and mitigating default risk. Ongoing monitoring and regulatory harmonization will ensure these reforms are sustainable, aligning Nigeria’s banking governance with global standards and supporting its role in economic growth (King & Levine, 1993; Siyanbola, 2019). This analysis sets the stage for further empirical exploration, as outlined in the research questions.

\subsection{1.1.2 Default Risk and Nigerian Banks}
The Nigerian banking sector, particularly deposit money banks (DMBs), has faced recurrent defaults, exposing critical weaknesses in corporate governance, risk management, and regulatory oversight. These failures have destabilized the financial system, eroded stakeholder confidence, and required substantial government interventions. The causes of bank defaults in Nigeria are multifaceted, stemming from a complex interplay of institutional, economic, political, supervisory, and regulatory inadequacies, as identified by Ogunleye (2002). A primary driver is weak corporate governance, marked by inadequate board oversight, lack of transparency, and poor management practices. The Central Bank of Nigeria (CBN) has highlighted inefficient, incompetent, and unstable management as a key factor in collapses, such as that of Savannah Bank, once a top 10 financial institution.

Governance failures often involve violations of best practices, including insider loans, unlawful credits, and transactions with shell companies controlled by bank insiders, amplifying moral hazard problems. Agency issues further heighten the risk of default. The principal-agent problem, first articulated by Adam Smith (1776), reveals how managers may prioritize self-interest over the institution’s survival, especially in banks with concentrated ownership structures. Fernando et al. (2019) note that companies with concentrated ownership show a higher correlation with default than those with distributed ownership, as managers misuse depositors’ funds through special purpose vehicles (SPVs) for unjust enrichment (Abubakar, 2011). This contributes to elevated non-performing loans (NPLs), which reached 37.3% in 2009—far above the global average of 6.6% (Nwosu, 2020)—reflecting poor risk assessment and lending practices. Economic and market dynamics also play a role. Informational asymmetries in loan markets lead to adverse selection, where high interest rates attract riskier borrowers, reducing the quality of loan portfolios and bank earnings (Stiglitz, 1987; Bertrand et al., 2011; Mayer & Moreno, 2003). In competitive environments, banks may lower interest rates to attract customers, inadvertently drawing in poor-quality borrowers and further weakening loan portfolios (Admati & Hellwig, 2013). These factors, combined with a lack of financial expertise on boards and inadequate risk management, create a fertile ground for default.

Several high-profile cases illustrate the severity of bank defaults in Nigeria. Savannah Bank’s collapse, as noted by the CBN, was driven by managerial incompetence and instability, marking a significant loss for a once-prominent institution. In 2009, five major banks—Afribank, Finbank, Intercontinental Bank, Oceanic Bank, and Union Bank—holding 30% of Nigeria’s deposits, faced imminent failure. The CBN intervened with a $3.9 billion injection to stabilize these institutions, underscoring the systemic risk posed by their distress. Afribank, one of Africa’s largest banks, saw its value plummet from $65 billion to a debt of N141 billion ($900,000), highlighting the scale of mismanagement and financial deterioration. The case of Oceanic Bank is particularly revealing. Mrs. Ibru and Mr. Oboden Ibru, board members, were implicated in governance violations, with 75% of the bank’s non-performing loans tied to enterprises they owned or controlled (Sanusi, 2009). Insider lending, unlawful credits, and asset acquisitions by family-owned shell companies violated corporate governance principles, while the CEO, Mrs. Ibru, awarded procurement contracts to her own shell companies on favorable terms. CBN Governor Sanusi (2009) emphasized that “weak corporate governance…at financial institutions were the root of the banking sector that almost collapsed in 2009,” a view supported by the Nigerian Deposit Insurance Corporation (NDIC), which executed insolvency activities on 425 financial institutions between 1988 and 2020 (Udo, 2020).

The repercussions of bank defaults in Nigeria extend beyond individual institutions, threatening financial stability and stakeholder interests. High levels of non-performing loans, peaking at 37.3% in 2009, eroded bank capital and liquidity, undermining their ability to meet depositor demands and fulfill lending roles (Nwosu, 2020). This loss of confidence among depositors, shareholders, and creditors destabilized the financial system, with ripple effects on the broader economy. The moral hazard problem, evident in cases like Oceanic Bank, saw depositors’ funds misused, shifting risk to taxpayers and necessitating costly bailouts, such as the $3.9 billion CBN intervention in 2009. Lack of transparency, including falsification of financial statements, further compounded the issue. This information asymmetry prevented stakeholders from making informed decisions, masking the true financial health of banks and delaying corrective action (Abubakar, 2011). The social burden was significant, as bank failures disrupted access to credit, hindered economic growth, and eroded public trust in the banking sector. Moreover, the prevalence of concentrated ownership and weak board oversight amplified agency conflicts, with managers prioritizing personal gain over sustainable growth, further jeopardizing depositors and the economy at large.

In response to these crises, the Nigerian government and the CBN have implemented several measures to bolster stability and governance in deposit money banks. In 2006, a consolidation strategy aimed to strengthen banks by merging weaker institutions, while in 2009, the CBN mandated a minimum capital base of 25 billion naira ($162 million) for insured banks (Onaolapo et al., 2019). These efforts sought to enhance resilience and liquidity, though challenges persisted. The CBN’s 2009 intervention, injecting $3.9 billion into ten banks and dismissing eight managing directors, addressed immediate liquidity crises but also exposed deep-seated governance failures (Fadare, 2011). Regulatory actions extended to enforcement, with the CBN removing 178 bank executives and blacklisting 75 for weak governance practices tied to agency issues (Jimoh & Iyoha, 2012). The NDIC played a critical role, managing insolvency for 425 institutions between 1988 and 2020, highlighting the scale of the challenge (Udo, 2020). Deposit insurance and prudential regulations, such as capital adequacy requirements and risk-based deposit insurance premiums, were introduced to protect depositors and curb risky behavior by bank owners reliant on government guarantees (Demirgüç-Kunt & Huizinga, 2010; World Bank, 2019). Despite these measures, regulatory multiplicity and inconsistent enforcement remain obstacles. Post-consolidation, many banks grew in size but suffered from poor management, ill-functioning board committees, and a lack of expertise, undermining governance mechanisms (Abubakar, 2011). The CBN and scholars like Kasum (2014) stress that corporate governance and agency issues were leading contributors to the 2009 crisis, necessitating ongoing monitoring and evaluation to ensure the effectiveness of these reforms in protecting depositors and promoting sustainability.

The recurring defaults in Nigerian deposit money banks highlight the urgent need for a comprehensive approach to corporate governance and regulation. Weak governance, agency conflicts, and inadequate risk management have fueled systemic instability, with high non-performing loans and moral hazard problems posing ongoing threats. While consolidation, capital requirements, and regulatory interventions have provided temporary relief, the persistence of supervisory gaps and the lack of board expertise demand further reform. Addressing these challenges requires strengthening board oversight, enhancing transparency, and aligning regulatory frameworks to mitigate risk, protect stakeholders, and ensure the long-term stability of Nigeria’s banking sector. This research contributes to these efforts by examining corporate governance practices to propose actionable reforms that enhance financial stability in Nigeria’s banking sector, directly informing RQ1 and RQ3 on governance indices and reform-induced shifts in NPL dynamics.

\subsection{1.1.3 Issues of Financial Regulatory Multiplicity in Nigeria}
Regulatory multiplicity in corporate governance pertains to the presence of various codes of conduct that govern the behaviors and responsibilities of stakeholders within an industry (John, 2023; Ofo, 2013; Adegbite, 2014). There is an imperative for enhanced clarity regarding the definition and implementation of corporate governance codes, which serve to delineate the expected conduct of employees in particular situations (Brooks, 1989). This issue is especially evident in Nigeria, where conflicting legal frameworks undermine the autonomy of the Board of Directors and the Audit Committee. Such conflict arises from the ownership structures of firms and contributes to inadequate accounting and internal governance practices, as well as overlapping and conflicting rules and responsibilities (Abdulmalik, 2020).

Moreover, the state of corporate governance in Nigeria is still developing, given that the nation has been sovereign for only 65 years as of 2025. Approximately forty percent of listed companies demonstrate a lack of understanding regarding the principles of corporate governance, a situation that can be traced back to the pre-colonial period in Nigeria (Nwaukwa, 2024; Oromaraghako et al., 2021; Wilson, 2006). This highlights the critical need for comprehensive reforms in regulatory frameworks to enhance the effectiveness of corporate governance in the country.

To effectively address the unique characteristics of a country, including cultural and political ideologies, it is important for countries to have corporate governance regulations that cater to their specific needs and reflect their colonial influences (Senaratne & Gunaratne, 2008; Lu et al., 2009; Ahunwan, 2002; Adegbite & Nakajima, 2011; Okike & Adegbite, 2012). Governments regulate companies, and companies are held legally accountable through their directors to shareholders and stakeholders through the concept of the 'corporate social contract' (Brooks, 1989). Cultural peculiarities and colonial influence have resulted in a lack of objective regulations in the Nigerian context, where banks are unable to facilitate the development of effective legislation in the banking industry (Ugochukwu, 2000).

Despite the increasing recognition of the need for effective corporate governance regulations, there is a lack of literature on the implications of regulatory multiplicity at the national level (Adegbite, 2012, 2014). On one hand, soft law promotes autonomy by encouraging voluntary compliance with good governance practices, while on the other hand, statutes carry a legal burden of responsibility for non-compliance (Adegbite, 2012, 2014). The argument for a statutory system of corporate governance in Nigeria is reinforced by the frequent failures of listed companies, linked to inadequate enforcement of regulations by regulatory agencies in addition to the saturation of regulations (Nwoke et al., 2023; Tayo-Tiwo, 2020). This position is substantiated by Wilson (2006), emphasizing the crucial role of strict adherence to fundamental principles of corporate governance.

The argument for an effective statutory system of corporate governance in Nigeria is further supported by the position of the Organization for Economic Cooperation and Development (OECD). The OECD posits that, in countries with weak judicial systems, such as Nigeria (Abdulmalik, 2020), independent regulatory enforcement may be more effective than judicial enforcement (Nwoke et al., 2023; Oman, 2001). This highlights the importance of regulatory agencies in promoting compliance with corporate governance principles and ensuring the accountability and transparency of listed companies. The issues of regulatory multiplicity continue to evolve with the development of new codes of conduct or revisions to existing regulations, exacerbating the proliferation of codes within Nigeria (Tayo-Tiwo, 2020; Abdulmalik, 2020; Adegbite, 2014).

However, the existence of multiple codes within a single country can address the specific needs of different firms and industries (Engle, 2007; Moghalu, 2011). Despite its potential benefits, this approach to corporate governance is also met with controversy. Some argue that multiple self-regulatory codes increase investor confidence (Macey & O’Hara, 2003; BCBS, 2005, 2006; Hoffmann, 1998), while others argue that it creates an environment of uncertainty and lack of cohesion among firms (Osemeke et al., 2016; Adegbite, 2014). Additionally, this uncertainty of applicable code can lead to weak enforcement and difficulty in distinguishing between firms that comply with good governance practices and those that do not, as is the case in Nigeria (Brooks, 1989; Abdulmalik, 2020; Wilson, 2006).

The Nigerian deposit money banking industry operates within a corporate governance framework that is both extensive and fragmented (Olojede et al., 2020). This landscape is characterized by the proliferation of multiple regulatory codes, which create a complex and often contradictory environment for financial institutions (Abdulmalik, 2020). The coexistence of various governance frameworks, each overseen by different regulatory bodies, has generated challenges that impinge on decision-making processes and the quality of corporate governance within the banking sector (Shaba, 2024). The Companies and Allied Matters Act (CAMA) 2020 serves as the cornerstone of corporate governance regulations in Nigeria, overseen by the Corporate Affairs Commission (CAC). As an overarching legal framework, CAMA governs company formation, board structures, shareholder rights, and compliance for registered entities. For example, it mandates the appointment of at least two directors (except for small companies), the organization of shareholder meetings, and the filing of annual reports (Muse, 2024). However, its general application to all companies often causes conflicts when it intersects with sector-specific codes, such as its provisions on director remuneration, which might contradict more stringent guidelines issued by regulators in the banking sector.

The Nigerian Code of Corporate Governance (NCCG) 2018, issued by the Financial Reporting Council of Nigeria (FRCN), is a principles-based framework. It adopts an "Apply and Explain" approach, requiring companies to either implement the code's principles or explain their non-compliance (Osemeke et al., 2016; Abdulmalik, 2020). While this approach theoretically provides flexibility, it also introduces ambiguities. Firms in the deposit money banking industry may exploit this by circumventing certain principles, such as board diversity or whistleblowing frameworks, when explanations of inapplicability suffice. The result is inconsistent governance practices (Nwaukwa, 2024). Moreover, since the NCCG does not override sector-specific codes, such as those issued by the Central Bank of Nigeria (CBN), banks are often forced to navigate overlapping and sometimes conflicting governance requirements (Tayo-Tiwo, 2020).

The CBN has issued strict and comprehensive codes to address governance in the financial sector, including the "Code of Corporate Governance for Banks and Discount Houses in Nigeria" (2014) and the "Code of Corporate Governance for Other Financial Institutions" (2019). These emphasize critical aspects such as board independence, risk management, and transparency—requirements tailored to address past challenges in the banking sector, such as the 2009 banking crisis (Iwedi et al., 2024). These codes are mandatory for banks and provide less room for flexibility compared to the NCCG (Tayo-Tiwo, 2020). However, conflicts arise because the CBN codes, while specific, may diverge from broader governance frameworks like the NCCG or other sectoral guidelines, for example, on risk management committee structures or diversity requirements (Odiase, 2024). Similarly, the Securities and Exchange Commission’s (SEC) Code of Corporate Governance for Public Companies (2011) introduces additional layers of governance aimed at protecting shareholder rights and ensuring board independence (Nwaukwa et al., 2024; Odiase, 2023). This code also conflicts at times with others, creating discrepancies that banks can exploit. For example, the SEC's stricter limits on interlocking directorships may clash with the CBN’s relatively lenient provisions, leaving room for selective adherence based on convenience or management preference. Regulatory fragmentation has its roots in Nigeria’s attempts to strengthen corporate governance following major crises, notably the 2009 banking sector collapse caused by unethical practices and weak oversight (Egbunike, 2023). While the intention has been to improve governance through multiple regulations tailored to specific issues, the resultant framework is disjointed and lacking in coordination. This has not only increased compliance costs but also slowed down decision-making processes. Furthermore, managers and boards exploit the gaps and overlaps between different codes, contributing to what has been identified in the literature as the agency problem—the prioritizing of the interests of managers over those of shareholders and other stakeholders (Kafidipe et al., 2021). The legal and disciplinary ambiguity inherent in regulatory multiplicity exacerbates this issue (Nwaukwa et al., 2024; Egbunike, 2023).

For instance, the NCCG's discretionary "Apply and Explain" approach allows companies to justify non-compliance with principles, such as whistleblowing or risk management, which are crucial for protecting stakeholder interests (Agbo, 2023). By contrast, the CBN’s mandatory compliance requirements impose stricter standards, but these often encourage companies to prioritize less stringent codes to manage their compliance risks. This flexibility has significant implications in contexts such as board independence, internal audit functions, and transparency (Ogunbameru & Alabi, 2023). For example, the lack of adequate whistleblowing mechanisms has been linked to the concealment of fraud and conflicts of interest, as highlighted during the 2009 crisis (Olojede et al., 2020; Nwoke et al., 2023). Weak enforcement compounds the problems associated with regulatory multiplicity. Despite the stringent provisions contained in CAMA, the NCCG, or the CBN codes, enforcement mechanisms remain inconsistent (Abdulmalik & Ahmad, 2020). The FRCN, tasked with monitoring compliance with the NCCG, lacks the authority to override sectoral regulators like the CBN, while shareholder associations lack the legal backing to enforce governance rules. According to Adegbite (2012), inadequate enforcement, combined with the fragmented nature of Nigeria’s regulatory system, weakens the credibility of the country’s corporate governance framework.

Another critical dimension to corporate governance in the Nigerian deposit money banking sector is the structure and independence of boards and audit committees (Osezua & Ude, 2023). The composition of these governance bodies is heavily influenced by ownership structures, often leading to limited independence. Research indicates that approximately 61% of bank boards in Nigeria lack nomination committees, despite such committees being essential for establishing board and management independence (Ogbechie, 2016). This is particularly concerning since global best practices, such as those enshrined in the Companies Act of 2008 in other jurisdictions, mandate the establishment of nomination committees to promote accountability and governance integrity. Empirical studies (Nwaukwa, 2024; Osemene et al., 2018; Puni, 2015) have shown that the absence of independent nomination committees in the Nigerian context is tied to concentrated ownership structures, which diminish overall board effectiveness. Furthermore, Section 359(4) of CAMA specifies that audit committees should consist of equal numbers of board directors and shareholders. However, as Adegbite (2014) reports, in practice, these committees often function at the behest of majority shareholders and managers. Their influence has been linked to the approval of financial statements without rigorous review, undermining the board's monitoring role. The former Governor of the Central Bank of Nigeria, Sanusi Lamido Sanusi, aptly summarized this issue, noting that many boards of Nigerian banks ignored corporate governance standards, failed to prevent insider dealings, and allowed undue influence by executives in decision-making processes (Sanusi, 2010).

To facilitate the development of an effective corporate governance framework in Nigeria, it is crucial to address the structural challenges posed by regulatory multiplicity, weak enforcement mechanisms, and excessive ownership influence. Existing literature emphasizes the significance of transparent accounting practices and robust regulatory enforcement in mitigating information asymmetry between managers, shareholders, and other stakeholders (Efunniyi et al., 2024; Yahaya et al., 2023; Ayogu, 2023; Salehi & Zimon, 2023; Karsono, 2023; Al-Shaer et al., 2022; Mason, 2020; Rezaee, 2003). These studies collectively argue that achieving a harmonized governance framework while empowering regulators to enforce uniform compliance is fundamental to aligning Nigeria’s corporate governance landscape with global best practices. In the absence of such reforms, systemic inefficiencies will persist, thereby undermining governance outcomes and the broader stability of the financial sector.

In addressing Nigeria’s fragmented governance framework, a concerted effort is needed to streamline and harmonize conflicting regulatory codes governing the deposit money banking industry. Collaboration among regulatory agencies is essential to establish a unified framework that promotes accountability while reducing the economic burden of compliance. Key structural reforms, such as the mandatory adoption of nomination committees and strengthened independence for both board and audit committees, are also critical to fostering transparency and accountability in the banking sector. Failure to implement these measures risks perpetuating the current challenges arising from regulatory multiplicity and agency problems, which in turn threaten the operational stability and growth prospects of Nigerian banks.

Conflicting regulatory codes remain a pressing issue for Nigerian deposit money banks, as they often present contradictions and create inconsistencies that have far-reaching consequences for stakeholders. This challenge emphasizes the urgent need for an in-depth review and proper implementation of governance codes to address these conflicts and reduce their potentially harmful impact on banking operations. For instance, Osemeke and Adegbite (2016) found that 67% of respondents in their research acknowledged significant disparities among Nigeria's regulatory codes, raising widespread concern about the effectiveness of the financial sector's governance structure. The same research also revealed that 80% of respondents believed the industry’s regulatory codes were overly compacted and conflicting, which underscores how regulatory complexity undermines the operational efficiency of deposit money banks. These findings underline the necessity of reevaluating and realigning Nigeria's regulatory framework to enhance its governance efficacy.

The persistence of conflicting regulatory frameworks not only disrupts the day-to-day operations of Nigerian deposit money banks but also poses systemic risks to their long-term stability and resilience. Addressing these challenges forms a key focus of this research, particularly in exploring how governance mechanisms—such as audit committee independence and regulatory clarity and enforcement consistency—influence the risk of bank default within the context of regulatory multiplicity in the Nigerian deposit money banking industry. These governance structures are vital tools for mitigating agency problems and reinforcing internal controls, which remain crucial within a financial sector characterized by overlapping and occasionally contradictory regulatory requirements. This analysis directly informs RQ4 on perceived regulatory clarity improvements and their causal impact on NPL declines.

% --- Table inserted here ---
\begin{table}[ht]
\centering
\begin{tabularx}{\textwidth}{|l|l|X|X|X|}
\hline
\textbf{Code} & \textbf{Governing Body} & \textbf{Key Provisions} & \textbf{Enforcement Approach} & \textbf{Conflicts with Other Codes} \\
\hline
CAMA 2020 & Corporate Affairs Commission (CAC) & Company formation, board structures, shareholder rights, annual reporting; mandates at least two directors and equal audit committee representation. & Statutory, mandatory with legal penalties. & General provisions on remuneration and directorships may contradict sector-specific CBN limits on tenure and independence. \\
\hline
NCCG 2018 & Financial Reporting Council of Nigeria (FRCN) & Principles-based; “Apply and Explain” for board diversity, risk management, whistleblowing. & Voluntary with explanations for non-compliance. & Flexibility allows circumvention of stricter CBN mandates on risk committees and diversity. \\
\hline
CBN Guidelines 2023 & Central Bank of Nigeria (CBN) & Board independence (≥50\% independent directors), risk management, CEO tenure limits (two 5-year terms). & Mandatory for banks, with sanctions for non-compliance. & Diverges from NCCG’s discretionary approach; stricter on interlocking directorships than SEC Code. \\
\hline
SEC Code 2011 & Securities and Exchange Commission (SEC) & Shareholder rights protection, board independence, limits on interlocking directorships. & Mandatory for public companies, with regulatory oversight. & Stricter directorship limits conflict with CBN’s more lenient provisions, enabling selective compliance. \\
\hline
\end{tabularx}
\caption{Corporate Governance Codes for Deposit Money Banks in Nigeria (2025)}
\label{tab:governancecodes}
\end{table}

\subsection{1.1.4 Bank Governance Challenges in Nigeria}

The historical evolution of corporate governance in Nigeria, particularly within the deposit money banking sector, is deeply intertwined with its pre-colonial and colonial legacies, highlighting a complex legal and institutional heritage that continues to shape its current challenges (Brennan & Solomon, 2008). Pre-colonial Nigeria consisted of distinct regional governance systems, such as those of the Hausa-Fulani, Yoruba, and Igbo, which operated through customary laws, monarchies, and communal structures. These frameworks, though informal by Western standards, were legitimate and effective systems for regulating trade, resolving disputes, and allocating resources (Altin, 2024; Elias, 2024). Businesses during this era were often family-owned or organized through community-based guilds, lacking the formal corporate governance structures seen today.

The colonial period (1861–1960) marked significant changes to Nigeria’s governance systems. British colonial administrators imposed foreign legal frameworks, such as the Companies Ordinance of 1912, modeled on the UK Companies Act (Agamah, 2024). These provisions were tailored to serve colonial enterprises like the Royal Niger Company and were exploitative in intent, prioritizing the colonial government’s agenda over the developmental needs of local businesses (Elias, 2024). They introduced Western corporate concepts, including limited liability and boards, but failed to integrate the socio-economic realities of Nigeria, disregarding indigenous governance systems and embedding a rigid legal template that persisted post-independence (Adegbite et al., 2012).

Following independence in 1960, Nigeria inherited this colonial legal infrastructure, which presented significant challenges in developing a governance system aligned with its economic and institutional realities (Elias, 2024). Instead of replacing colonial laws entirely, policymakers amended and expanded the existing framework, as seen in the Companies Act of 1968 (later replaced by the Companies and Allied Matters Act, 2020, or CAMA). These adaptations aimed to modernize the governance structure to support a growing economy and attract foreign investment. However, gaps in institutional capacity and the necessity for legal continuity limited progress, impeding the creation of a cohesive governance model suited to the Nigerian context (Udo, 2020). As Nigeria’s economy grew, particularly in sectors like oil, telecommunications, and banking, regulatory bodies such as the Central Bank of Nigeria (CBN), the Securities and Exchange Commission (SEC), and the Nigerian Communications Commission (NCC) emerged. These institutions sought to address sector-specific issues by introducing additional governance codes. However, rather than replacing the colonial legal foundation, newer frameworks were layered on top, resulting in conflicting and overlapping regulations. For example, Section 279 of CAMA outlines general director duties, which can conflict with the more detailed governance requirements of the CBN Code, creating ambiguity for banks operating under these intersecting guidelines (Osemeke & Adegbite, 2016).

This history of regulatory multiplicity contributes to ongoing governance challenges in the Nigerian deposit money banking system (Babajide et al., 2020). Cases of fraud and mismanagement have highlighted the inadequacies of current governance structures. Despite regulatory instruments such as the CBN Act, the Banks and Other Financial Institutions Act (BOFIA), and the Nigeria Deposit Insurance Corporation (NDIC) Act, enforcement remains weak (Nwoke et al., 2023). The CBN Act of 2007 grants the Central Bank autonomy to oversee financial institutions, while BOFIA ensures regulatory powers over banking operations (Obadire et al., 2022). However, enforcement of these laws is undermined by institutional weaknesses, such as outdated sanctions and resource constraints, as noted in the ROSC (2004) report (Azoro et al., 2022). The conflicts between regulatory frameworks exacerbate governance challenges for Nigerian deposit money banks. For example, while the CBN Code explicitly prohibits family members from simultaneously occupying key board positions such as Chairman and CEO, the SEC Code (2011) does not impose similar limitations, creating regulatory inconsistencies. This misalignment fosters opportunities for regulatory arbitrage, enabling banks to selectively comply with mandatory codes like the CBN’s while neglecting voluntary recommendations from the SEC. Consequently, such contradictions weaken the impact of governance reforms and undermine institutional accountability within the banking sector (Osemeke & Adegbite, 2016).

The relationship between board structure and organizational performance in the Nigerian deposit money banking industry further complicates reform efforts. Globally, empirical studies have yielded mixed findings on whether board size and complexity correlate with organizational performance. While some studies indicate no significant relationship (Romano et al., 2012; Al Hawary, 2011), others suggest varying outcomes based on institutional and cultural contexts (Adams & Mehran, 2003; Trabelsi, 2010). In Nigeria, governance obstacles such as concentrated ownership and unqualified board members further undermine the potential for effective oversight, even when formal governance frameworks are in place.

Enforcement gaps across regulatory institutions amplify these challenges. Reports such as the ROSC (2004) indicate that the CBN lacks sufficient capacity to ensure compliance with financial reporting standards, while the SEC struggles with effective oversight within the capital markets. These structural deficiencies have allowed instances of governance failure, such as mismanagement and financial impropriety, to persist in the banking sector, eroding trust and stability (Nwoke, 2023).

To address these critical issues, it is imperative to unify and harmonize Nigeria’s regulatory frameworks while bolstering enforcement capabilities. Streamlining governance codes would reduce ambiguities and foster coherence in regulatory oversight (Osemeke & Adegbite, 2016). Furthermore, research into the connection between board composition and performance in Nigerian deposit money banks could offer evidence-based insights into effective governance practices. Strengthening institutional capacity and aligning governance practices with the local context are essential to improving transparency, accountability, and overall governance outcomes in the financial sector. Corporate governance challenges in Nigeria’s deposit money banking industry are deeply rooted in the historical legacy of colonial legal systems, compounded by regulatory overlaps and institutional inefficiencies. While reforms have been introduced to improve governance structures, the persistence of conflicting regulations and weak enforcement undermines their effectiveness. A consistent and localized approach to governance, coupled with stronger regulatory enforcement, is vital to ensuring the long-term sustainability and resilience of Nigerian deposit money banks. This research particularly examines how governance mechanisms, such as audit committee independence and management qualifications, interact with regulatory clarity to influence non-performing loan (NPL) ratios, providing insights into mitigating default risk in the context of Nigeria’s fragmented regulatory environment, thereby advancing RQ2 and RQ4.

This backdrop necessitates a methodical investigation into the interdependencies between corporate governance mechanisms and NPL dynamics, as articulated in the research questions that guide this thesis. By examining these elements through a mixed-methods lens—encompassing panel fixed-effects on 160 bank-years and bootstrapped mediation on 234 stakeholder perceptions—the study aims to yield practicable insights for enhancing governance efficacy in Nigeria's DMB sector.

\section{1.2 Problem Statement}
% [Problem statement converted]
Despite successive regulatory interventions in Nigeria's deposit money banking (DMB) sector, including the Central Bank of Nigeria's (CBN) 2014 Code of Corporate Governance, the Nigerian Code of Corporate Governance (NCCG) 2018, the Companies and Allied Matters Act (CAMA) 2020, and the CBN's 2023 Guidelines on Board Composition and Tenure, non-performing loan (NPL) ratios persist at elevated levels, underscoring a critical empirical gap. Historical data reveal recurrent crises, with NPL ratios exceeding 37% during the 2009 financial upheaval and contributing to over 425 institutional failures since 1988, even as post-consolidation reforms aimed to fortify capital bases and risk management (Sanusi, 2014; Olojede et al., 2020). Although NPL ratios have moderated to approximately 5.62% by April 2024, they remain susceptible to macroeconomic volatilities such as oil price fluctuations, inflation rates surpassing 24%, and currency devaluations, which erode lending portfolios and amplify default risks (CBN, 2024; IMF, 2024). This persistence indicates that existing reforms have not sufficiently attenuated systemic vulnerabilities, necessitating a deeper examination of governance mechanisms' efficacy in curbing credit impairments.

On the practical front, while governance frameworks mandate robust board structures and disclosure protocols, enforcement remains fragmented and ineffective, particularly in ensuring audit committee independence and operational adherence. Audit committees, intended to safeguard financial reporting integrity, often lack genuine autonomy due to insider dominance and familial entanglements, which perpetuate opacity and enable asset misclassification (Adegbite, 2014; Abdulmalik & Ahmad, 2020). Consequently, practices such as whistleblowing and risk oversight falter, allowing non-performing assets to accumulate unchecked amid weak internal controls. This gap is exacerbated in an environment where compliance indices, though formally adopted, are undermined by cultural norms favoring relational networks over meritocratic accountability, thereby hindering the translation of policy intent into tangible risk mitigation (Kafidipe, 2021).

Theoretically, extant literature predominantly isolates singular governance elements—such as board composition or ownership structures—without synthesizing the interplay among disclosure, practice, and compliance indices, thus overlooking their collective modulation of NPL dynamics. Agency theory posits that misaligned incentives precipitate moral hazards, yet few studies integrate stewardship theory's emphasis on intrinsic motivations or institutional theory's focus on regulatory embeddedness to explain governance failures in emerging markets (Jensen & Meckling, 1976; Davis et al., 1997; Scott, 2014). Signalling theory further suggests that enhanced disclosures could convey credibility to stakeholders, but empirical validations in Nigeria remain sparse, particularly post-2014, where reforms aligned with Basel III and OECD principles have theoretically bolstered transparency without commensurate theoretical reconciliation (OECD, 2023; BCBS, 2010). This fragmentation limits a holistic understanding of how governance apparatuses attenuate default risks in oil-dependent economies.

Methodologically, while quantitative analyses dominate, they often rely on cross-sectional designs or simple regressions, neglecting mixed-methods approaches that combine longitudinal panel data with perceptual surveys and bootstrapped mediation models to unpack causal pathways. For instance, prior research employs ordinary least squares without fixed effects to control for unobserved heterogeneities, thereby inflating spurious correlations amid Nigeria's volatile macroeconomic context (Osita et al., 2019; Natufe & Evbayiro-Osagie, 2023). The scarcity of studies incorporating Hayes' Process Model 4 for mediation or difference-in-differences frameworks to assess reform impacts further constrains inferences on governance's temporal evolution, especially in integrating stakeholder perceptions from 234 respondents across 10 DMBs with 160 bank-year observations (Hayes, 2018; Baltagi, 2021).

Regulatory multiplicity compounds these issues, as overlapping mandates from the CBN, Nigeria Deposit Insurance Corporation (NDIC), Securities and Exchange Commission (SEC), and Financial Reporting Council of Nigeria (FRCN) engender compliance ambiguities and enforcement inconsistencies. This fragmentation, rooted in institutional silos, dilutes accountability and fosters arbitrage opportunities, where banks navigate divergent guidelines—such as CBN's executive vetting versus SEC's disclosure requirements—leading to suboptimal governance outcomes (Adegboye, 2020; Udo, 2020). Comparative insights from Ghana and South Africa highlight that harmonized regimes enhance clarity, yet Nigeria's post-2020 efforts remain nascent, perpetuating inefficiencies amid economic shocks.

Collectively, these gaps culminate in persistent credit fragility, eroding stakeholder confidence, constraining lending activities, and impeding broader economic growth in a sector pivotal to Nigeria's GDP. These gaps collectively necessitate an integrated inquiry into how governance mechanisms and regulatory coherence shape NPL ratios in Nigerian DMBs.

These objectives collectively operationalize the thesis's mixed-methods architecture, integrating longitudinal quantitative analysis with cross-sectional perceptual insights to yield robust, practicable recommendations for governance unification and reform efficacy in Nigeria's banking sector.
\section{1.3 Research Aim and Objectives}
\subsection{Aim}
The overarching aim of this thesis is to examine how corporate governance indices and regulatory clarity affect non-performing loan ratios in Nigerian deposit money banks between 2009 and 2025, thereby disentangling the interdependencies between governance mechanisms, institutional reforms, and default risk amid persistent sectoral vulnerabilities.
\subsection{Objectives}
\begin{enumerate}
  \item To examine the extent to which corporate governance indices—encompassing disclosure via the Corporate Governance Disclosure Index, practice through the Practice Index, and compliance with the Compliance Index—modulate non-performing loan ratios in Nigerian deposit money banks from 2009 to 2024, while controlling for bank size, capital adequacy, and macroeconomic factors such as GDP growth, inflation, and oil prices.
  \item To evaluate the degree to which audit committee independence reduces 2024 non-performing loan ratios in Nigerian deposit money banks, and to assess whether this relationship is partially mediated by enhanced financial reporting integrity as perceived by bank insiders.
  \item To determine whether the linkage between corporate governance disclosure and non-performing loan ratios shifts from statistically insignificant in the pre-reform period (2009–2014) to significantly negative in the post-reform period (2016–2024), attributable to Central Bank of Nigeria mandates that enhance disclosure credibility through board tenure limits, at least 50\% independent directors, and executive vetting.
  \item To assess the extent to which perceived regulatory clarity improvements from Central Bank of Nigeria and Securities and Exchange Commission harmonization (2020–2025) cause declines in non-performing loan ratios (\(\beta < 0, p < .05\)), while controlling for economic shocks such as devaluation and inflation.
\end{enumerate}

\section{1.4 Research Questions}
\begin{enumerate}
  \item To what extent do corporate governance indices—encompassing disclosure via the Corporate Governance Disclosure Index, practice through the Practice Index, and compliance with the Compliance Index—modulate NPL ratios in Nigerian DMBs from 2009 to 2024, controlling for bank size, capital adequacy, and macroeconomic factors like GDP growth, inflation, and oil prices?
  \item To what degree does audit committee independence reduce 2024 NPL ratios in Nigerian DMBs, and is this relationship partially mediated by enhanced financial reporting integrity as perceived by bank insiders?
  \item Does the linkage between corporate governance disclosure and NPL ratios shift from statistically insignificant pre-reform (2009–2014) to significantly negative post-reform (2016–2024), attributable to CBN mandates that enhance disclosure credibility through board tenure limits, \(\geq 50\%\) independent directors, and executive vetting?
  \item To what extent do perceived regulatory clarity improvements from CBN and SEC harmonization (2020–2025) cause NPL ratio declines (\(\beta < 0, p < .05\)), controlling for economic shocks such as devaluation and inflation?
\end{enumerate}

\section{1.5 Theoretical Framework}
% [Agency, stewardship, institutional, signalling theory subsections converted]
This section delineates the conceptual underpinnings that scaffold the inquiry into corporate governance mechanisms and their ramifications for non-performing loan (NPL) ratios in Nigerian deposit money banks (DMBs). Drawing upon a synthesis of agency theory, stewardship theory, institutional theory, and signalling theory, the framework elucidates the interdependencies between governance apparatuses, regulatory reforms, and default risk mitigation. These lenses are strategically mapped to the research questions (RQs) to furnish a theoretically robust architecture: RQ1 interrogates the modulating effects of governance indices through agency and institutional prisms; RQ2 examines audit committee independence and mediation via agency and stewardship perspectives; RQ3 scrutinizes reform-induced shifts in disclosure efficacy under institutional and signalling theories; and RQ4 probes regulatory clarity enhancements informed by institutional and agency paradigms. This integration not only anchors the empirical analysis but also illuminates pathways for theoretical extension in emerging market contexts, where institutional voids and regulatory fragmentation amplify governance challenges.

\subsection{1.5.1 Agency Theory}
Agency theory posits that conflicts arise between principals (shareholders and depositors) and agents (managers and board members) due to divergent interests, information asymmetries, and opportunistic behaviors, necessitating monitoring mechanisms to align incentives and attenuate moral hazard (Jensen & Meckling, 1976). In banking contexts, such conflicts manifest in insider lending, excessive risk-taking, and asset misallocation, which erode capital adequacy and precipitate systemic vulnerabilities. Effective governance—encompassing board oversight, disclosure mandates, and compliance protocols—serves as a corrective apparatus, curbing agency costs through contractual safeguards and incentive structures.

Applied to Nigerian DMBs, agency theory elucidates the persistent governance erosions that have fueled recurrent crises, such as the 2009 upheaval where NPL ratios exceeded 37%, driven by unchecked insider abuses and weak board monitoring (Sanusi, 2014). For instance, inadequate separation of ownership and control has enabled managerial entrenchment, exacerbating default risks amid macroeconomic volatilities like oil price fluctuations. Similar patterns are observed in Ghana and Kenya, where agency-driven insider lending has similarly inflated NPLs, underscoring the need for robust monitoring indices to mitigate such pathologies (Osei, 2021; Adegbite, 2014). This theory undergirds RQ1 by framing governance indices (disclosure, practice, and compliance) as agency-cost reducers, controlling for confounders like bank size and inflation, and RQ4 by highlighting how regulatory harmonization between the Central Bank of Nigeria (CBN) and the Securities and Exchange Commission (SEC) can enforce accountability, yielding NPL declines (β < 0, p < .05).

\subsection{1.5.2 Stewardship Theory}
Stewardship theory counters agency assumptions by conceptualizing managers as intrinsically motivated stewards who prioritize organizational welfare over self-interest when empowered through trust, autonomy, and long-term alignment (Davis et al., 1997). Rather than adversarial monitoring, this perspective advocates for collaborative governance structures—such as independent audit committees and empowered boards—that foster psychological ownership and ethical decision-making, thereby enhancing operational integrity and risk management.

In the Nigerian banking milieu, stewardship theory illuminates how empowered audit committees can cultivate a culture of integrity, particularly in post-reform eras marked by CBN mandates for ≥50% independent directors and executive vetting (CBN, 2014). For example, where agency conflicts dominate due to historical nepotism and weak enforcement, stewardship mechanisms like enhanced financial reporting can mediate reductions in NPL ratios by promoting proactive risk stewardship amid economic shocks (Abdulmalik & Ahmad, 2020). Comparative evidence from South Africa reveals analogous stewardship dynamics, where trusted audit autonomy has mitigated NPL escalations during fiscal downturns, contrasting with Nigeria's more coercive regulatory landscape (Olojede et al., 2020). This theory informs RQ2, positing that audit committee independence attenuates 2024 NPL ratios via mediated financial reporting integrity, as perceived by insiders, thereby challenging agency-centric views in contexts of institutional reform.

\subsection{1.5.3 Institutional Theory}
Institutional theory explicates organizational behaviors as responses to external pressures—coercive (regulatory mandates), mimetic (imitation of successful peers), and normative (professional standards)—that shape governance practices and legitimize reforms (DiMaggio & Powell, 1983). In emerging economies, institutional voids, such as regulatory fragmentation, compel firms to adopt isomorphic strategies to secure legitimacy and operational stability, particularly in sectors prone to crises.

For Nigerian DMBs, this theory contextualizes the evolution from pre-reform governance inertia (2009–2014) to post-reform activation (2016–2024), driven by coercive forces like the CBN Code of 2014, the Nigerian Code of Corporate Governance (NCCG) 2018, the Companies and Allied Matters Act (CAMA) 2020, and CBN Guidelines 2023, which impose board tenure limits and disclosure enhancements (Natufe & Evbayiro-Osagie, 2023). These pressures have prompted mimetic adoption of Basel III standards, reducing NPL ratios through normative professionalization. Analogous institutional dynamics in OECD nations, such as post-2008 reforms in the UK, demonstrate how coercive harmonization attenuates default risks, offering benchmarks for Nigeria's fragmented ecosystem (OECD, 2023). Institutional theory anchors RQ1 by integrating macroeconomic controls into governance index effects, RQ3 through a difference-in-differences framing of reform shifts in disclosure-NPL linkages, and RQ4 by attributing NPL declines to perceived clarity improvements from CBN-SEC harmonization (2020–2025), controlling for devaluation and inflation.

\subsection{1.5.4 Signalling Theory}
Signalling theory asserts that firms convey credible information to stakeholders via observable actions, such as disclosures, to mitigate information asymmetries and enhance market perceptions, thereby reducing adverse selection and funding costs (Spence, 1973). In banking, robust disclosure signals governance quality, fostering investor confidence and disciplining risk behaviors.

Within Nigerian DMBs, signalling theory explains how enhanced disclosure post-reforms—mandated by CBN vetting and independent directors—transforms insignificant pre-reform effects into significantly negative NPL associations, as stakeholders interpret transparency as a proxy for reduced agency risks (Adegboye, 2020). This is evident in the 2009 crisis, where opaque reporting amplified asymmetries, leading to over 425 institutional failures since 1988. Comparative insights from emerging markets like India highlight how signalling via disclosure indices has curbed NPLs during reforms, paralleling Nigeria's trajectory toward Basel III convergence (Kafidipe, 2021). The theory underpins RQ3, framing disclosure as a reform-activated signal that shifts NPL dynamics, thereby complementing institutional pressures in yielding practicable insights for regulatory unification.

Building upon this foundation, the theoretical framework synthesizes these lenses to hypothesize governance-NPL interdependencies, tested via mixed-methods: longitudinal fixed-effects for RQ1 and RQ3, and cross-sectional mediation for RQ2 and RQ4. This approach not only corroborates but extends theories by embedding them in Nigeria's reform-laden context, contributing to institutional theory through evidence of coercive-mimetic synergies and to agency-stewardship debates via perceptual mediations.


\section{1.6 Significance of the Study}
% [Converted content]
This study contributes by addressing entrenched gaps in the corporate governance literature, particularly within emerging market banking sectors like Nigeria's deposit money banks (DMBs), where recurrent crises—such as the 2009 upheaval with non-performing loan (NPL) ratios exceeding 37% and over 425 institutional failures since 1988—underscore the need for theoretically robust and empirically grounded analyses. Theoretically, it advances agency theory by disentangling how governance mechanisms mitigate principal-agent conflicts through enhanced disclosure, practice, and compliance indices, while integrating stewardship theory to elucidate audit committee independence as a mediator of financial reporting integrity. Correspondingly, it enriches institutional theory by examining regulatory harmonization's role in activating governance reforms, thereby challenging extant assumptions of uniform efficacy across contexts and synthesizing comparative insights from Ghana and South Africa, where similar post-reform trajectories reveal context-specific institutional pressures. This theoretical synthesis not only fills voids in mixed-methods governance scholarship but also refines signalling theory by demonstrating how post-2014 CBN mandates—such as board tenure limits and ≥50% independent directors—signal credibility to stakeholders, attenuating asymmetric information and precipitating NPL declines.

Methodologically, this study contributes by pioneering a pragmatic mixed-methods architecture that amalgamates longitudinal quantitative panel fixed-effects models across 160 bank-years (2009–2024) with cross-sectional blended approaches, including bootstrapped mediation via Hayes Process Model 4 on 234 stakeholder perceptions and difference-in-differences framing to isolate reform impacts. This innovation transcends conventional single-method limitations, offering a replicable framework for volatile emerging economies where macroeconomic confounders like oil price fluctuations and inflation necessitate controlled, multi-faceted reasoning. By triangulating audited financial data, CBN reports, and semi-structured interviews from 10 DMBs, it enhances methodological rigor, providing a blueprint for future econometric inquiries into governance-NPL interdependencies that balances positivist precision with interpretivist depth.

From a policy perspective, this study contributes by furnishing actionable insights for regulators such as the Central Bank of Nigeria (CBN), Securities and Exchange Commission (SEC), and Nigeria Deposit Insurance Corporation (NDIC), advocating unified governance frameworks that harmonize fragmented mandates under CAMA 2020, NCCG 2018, and CBN Guidelines 2023. Building upon the findings that regulatory clarity improvements (2020–2025) yield significant NPL reductions (β < 0, p < .05), it proposes enforceable mechanisms—such as streamlined compliance protocols and executive vetting—to mitigate regulatory multiplicity, thereby aligning Nigerian practices with Basel III and OECD standards and fostering systemic resilience amid economic shocks like devaluation.

Managerially, this study contributes by equipping DMB boards and audit committees with empirical evidence on governance indices' modulating effects on NPL ratios, emphasizing audit independence's partial mediation through financial reporting integrity. This enables practitioners to fortify internal controls, operational efficiency, and risk management, as evidenced by the shift from insignificant pre-reform disclosure impacts (2009–2014) to significantly negative post-reform associations (2016–2024), thus empowering strategic reforms that enhance capital adequacy and stakeholder trust in a sector prone to governance erosions.

Societally, this study contributes by bolstering financial stability and investor confidence in Nigeria's DMB sector, which underpins broader economic development in a nation where banking fragility has historically exacerbated inequality and impeded growth. By illuminating pathways to NPL abatement—controlling for GDP growth, inflation, and bank size—it promotes inclusive financial access, safeguards depositor interests, and mitigates crisis spillovers, ultimately advancing sustainable development goals in sub-Saharan Africa through evidence-based governance enhancements.### 1.7 Scope and Delimitations

This thesis delineates its scope and delimitations to ensure a focused inquiry into corporate governance mechanisms within Nigeria's deposit money banking (DMB) sector, emphasizing their interplay with reforms and impacts on non-performing loan (NPL) ratios. By establishing precise boundaries, the study maintains methodological rigor while addressing the four research questions that probe governance indices, audit committee independence, reform-induced shifts in disclosure effects, and regulatory clarity enhancements (as articulated in Section 1.3). These parameters are calibrated to the post-2009 crisis landscape, where recurrent institutional failures—exemplified by NPL ratios exceeding 37% in 2009 and over 425 bank collapses since 1988—underscore the urgency of targeted analysis amid regulatory evolutions like the CBN Code of 2014, NCCG 2018, CAMA 2020, and CBN Guidelines 2023 (Sanusi, 2014; Olojede et al., 2020; OECD, 2023).


\section{1.7.1 Scope and Delimitations of Study}
The scope encompasses a multifaceted examination of corporate governance apparatuses in Nigerian DMBs, operationalized through a mixed-methods architecture that integrates longitudinal quantitative data with cross-sectional perceptual insights. Correspondingly, the population is confined to 10 Tier-1 DMBs, selected for their systemic significance and representation of approximately 70% of the sector's asset base, as per CBN classifications (CBN, 2024). This focus on major institutions—such as Access Bank, Zenith Bank, and United Bank for Africa—facilitates robust econometric modeling of governance effects, controlling for heterogeneities in bank size and capital adequacy, while yielding practicable insights into default risk attenuation. Building upon this foundation, the temporal horizon spans 2009 to 2025, capturing the post-crisis epoch from the 2009 upheaval to the latest reform trajectories, including Basel III convergence and CBN's 2023 guidelines on executive vetting and board independence (BCBS, 2010; Adegbite, 2014). This periodization enables difference-in-differences analyses for RQ3, assessing pre-reform (2009–2014) insignificance versus post-reform (2016–2024) negative linkages between disclosure and NPL ratios, amid macroeconomic volatilities like oil price fluctuations and inflation spikes.

Data types further define the scope: panel financials from audited statements and CBN reports yield 160 bank-years for fixed-effects regressions in RQ1 and RQ3, quantifying governance indices' modulation of NPL ratios while incorporating controls such as GDP growth and devaluation shocks. This is supplemented by a 2024 cross-sectional survey of 234 stakeholders across the 10 DMBs, employing bootstrapped mediation (Hayes Process Model 4) to dissect RQ2's mediation via financial reporting integrity and RQ4's causal pathways from regulatory harmonization to NPL declines (β < 0, p < .05). Semi-structured interviews with 10 management-level participants provide qualitative depth, illuminating interdependencies between governance erosions and institutional activations, thus bridging agency theory's principal-agent conflicts with stewardship theory's collaborative ethos in an emerging market context (Donaldson & Davis, 1991; Jensen & Meckling, 1976). These elements collectively advance theoretical synthesis, contrasting institutional theory's emphasis on regulatory fragmentation with signalling theory's disclosure credibility enhancements, while contextualizing Nigerian reforms within global benchmarks like South Africa's King IV Code and Ghana's Banking Act amendments (Institute of Directors Southern Africa, 2016; Bank of Ghana, 2018).

\subsection{Deliminations of Study}
Delimitations are intentionally imposed to sharpen analytical precision and empirical feasibility, excluding extraneous elements that could dilute the study's scholarly depth. Notably, the inquiry excludes microfinance and merchant banks, prioritizing DMBs due to their centrality in systemic risk propagation and alignment with the research questions' focus on deposit-taking institutions vulnerable to NPL escalations under fragmented oversight (NDIC, 2023). This boundary is appropriate, as microfinance entities operate under distinct regulatory regimes (e.g., CBN's Microfinance Policy Framework), rendering their inclusion incompatible with the panel fixed-effects methodology tailored to DMB-specific indices like the Corporate Governance Disclosure Index (CGDI). Furthermore, qualitative respondents are limited to management-level personnel—such as executives, risk managers, and board members—to capture informed perceptions of governance mechanisms, eschewing frontline staff whose insights, while valuable, may lack strategic oversight on audit independence or compliance efficacy (Kafidipe, 2021). This delimitation ensures thematic saturation in interviews, aligning with RQ2 and RQ4's emphasis on insider-mediated relationships and regulatory clarity perceptions.

Additionally, the study focuses exclusively on internal governance mechanisms—disclosure, practice, and compliance—rather than macro policy design, such as fiscal interventions or monetary authority restructuring. This restriction is justified by the thesis's theoretical grounding in institutional theory, which posits that firm-level apparatuses, activated through reforms like ≥50% independent directors and board tenure limits, precipitate NPL attenuations without necessitating broader economic policy critiques (North, 1990; Scott, 2014). By concentrating on these mechanisms, the analysis avoids overextension into exogenous domains like geopolitical oil dependencies, thereby enhancing econometric robustness in controlling for macroeconomic confounders. These delimitations ensure analytical depth while maintaining empirical manageability.


% --- end of Chapter 1 content ---

\cleardoublepage
\printbibliography[heading=bibintoc]

\end{document}
