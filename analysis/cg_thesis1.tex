% --------------------------------------------------------------
%  The University of Buckingham – Thesis Paper
%  FULLY COMPLIANT + WORKS ON ARCH LINUX
% --------------------------------------------------------------
\documentclass[12pt,a4paper]{report}
\usepackage[margin=20mm]{geometry}

% --- MODERN TIMES FONT (replaces times.sty) ---
\usepackage{newtxtext,newtxmath}

\usepackage{setspace}
\onehalfspacing

\usepackage{graphicx}
\usepackage{booktabs}
\usepackage{array}
\usepackage{amsmath,amssymb}

% --- NATBIB + APALIKE (APA-style, built-in) ---
\usepackage[authoryear,round]{natbib}
\bibliographystyle{apalike}

\usepackage{tocloft}
\usepackage{fancyhdr}
\usepackage{caption}
\usepackage{hyperref}
\usepackage{enumitem}
\setlist[enumerate]{label=\alph*),leftmargin=*}
\setlist[itemize]{leftmargin=*}

\pagenumbering{roman}
\pagestyle{plain}

\title{Corporate Governance Transparency and Non-Performing Loans:\\
       Evidence from Nigerian Banks Before and After CBN Reforms}
\author{Isaiah Jimoh-Ibrahim\\
        211828@buckingham.ac.uk\\
        The University of Buckingham,\\
        Hunter Street, Buckingham MK18 1EG, UK}
\date{October 2025}

\begin{document}

\maketitle
\thispagestyle{empty}
\newpage

\begin{abstract}
This paper examines the impact of corporate governance transparency on
non-performing loans (NPLs) in Nigerian banks, exploiting the Central Bank
of Nigeria’s (CBN) 2014--2016 reforms as a natural experiment. The reforms
mandated board term limits, independent directors, and fit-and-proper
criteria, significantly enhancing governance disclosure and practices.
Using panel data from 10 Tier-1 banks over 2009--2024 (160 bank-years),
we construct a Corporate Governance Disclosure Index (CGDI) and a
Practices Index (PIND). Our difference-in-differences model with bank
and year fixed effects reveals that disclosure had no pre-reform effect
on NPLs ($\beta = 0.046$, $p = 0.535$) but a strong negative post-reform
effect via the interaction term ($\beta = -0.259$, $p = 0.010$),
yielding a total post-reform effect of $-0.213$. Practices consistently
reduce NPLs ($\beta = -0.582$, $p < 0.001$). A 10-point CGDI increase
post-reform lowers NPLs by 2.1 percentage points (30\% relative to the
pre-reform mean). Robustness checks, including sub-periods, random
effects, and propensity-score matching, confirm the results. We
formalize and test two hypotheses: (H$_1$) disclosure reduces NPLs only
post-reform due to institutional activation; (H$_2$) governance
practices dominate disclosure throughout. Policy implications for
emerging markets emphasize enforceable practices over voluntary
disclosure.

\textbf{JEL Classifications:} G21, G28, G34, O16, O55, C23, D82  
\textbf{Keywords:} Corporate governance, Non-performing loans,
Disclosure, Nigeria, Bank reforms, Fixed effects
\end{abstract}
\newpage

\section*{Declaration of Originality and Prior Publication}
I declare that this work is my own and has not been submitted for a degree
at this or any other university. No part of the thesis has been
previously published except where full bibliographic details are given
in the text and bibliography. All pre-published material has been
revised and integrated into the overall argument of the thesis.

\vspace{2em}
\rule{8cm}{0.4pt}\\Isaiah Jimoh-Ibrahim\\
Date: October 30, 2025
\newpage

\tableofcontents
\newpage

\pagenumbering{arabic}
\pagestyle{fancy}
\fancyhf{}
\fancyhead[LE,RO]{\thepage}
\fancyhead[LO]{\nouppercase\leftmark}
\fancyhead[RE]{\nouppercase\rightmark}

\chapter{Introduction}
Non-performing loans (NPLs) pose a persistent threat to financial
stability, particularly in emerging markets where they averaged 27.6\% of
total loans in Nigeria in 2009 \citep{ozili2020}. By 2024 this ratio had
plummeted to 3.9\%, coinciding with Central Bank of Nigeria (CBN)
reforms in 2014--2016. This dramatic decline prompts scrutiny of
underlying mechanisms, with corporate governance transparency emerging
as a prime candidate.

Agency theory posits that opaque governance exacerbates moral hazard and
adverse selection, inflating NPLs \citep{jensen1976}. \citet{bushman2003}
formalize disclosure’s role in mitigating information asymmetries, yet
empirical evidence is mixed: beneficial in developed markets
\citep{gompers2003}, insignificant or perverse in emerging markets
\citep{klapper2004}. Institutional voids—weak enforcement and investor
protection—may render disclosure inert \citep{laporta1998}.

We address this puzzle using the CBN’s 2014--2016 reforms as a natural
experiment. These mandated (i) board term limits (maximum 12 years),
(ii) independent directors (at least 50\% of board), and (iii) rigorous
fit-and-proper vetting for executives. Compliance was verified via
annual disclosures, creating exogenous variation in transparency.

Prior studies overlook Nigeria—Africa’s largest economy—and conflate
disclosure with practices \citep{ozili2020,gaganis2018}. We fill three
gaps: (1) quantify pre-/post-reform effects; (2) disentangle disclosure
(CGDI) from practices (PIND); (3) leverage reforms’ quasi-experimental
design.

Our sample comprises 10 Tier-1 Nigerian banks over 2009--2024
(\(N=160\) bank-years). The baseline model is:

\begin{equation}
\label{eq:baseline}
\text{NPLR}_{it} = \beta_0 + \beta_1 \text{CGDI}_{it}
                 + \beta_2 (\text{CGDI}_{it} \times \text{Post2015}_t)
                 + \beta_3 \text{PIND}_{it}
                 + \boldsymbol{\gamma}'\mathbf{X}_{it}
                 + \alpha_i + \delta_t + \varepsilon_{it},
\end{equation}

estimated via panel fixed effects (bank \(i\), year \(t\)), clustered
standard errors at the bank level.

We find disclosure ineffective pre-reform (\(\beta_1 = 0.046\),
\(p = 0.535\)) but strongly negative post-reform (\(\beta_2 = -0.259\),
\(p = 0.010\); total \(\beta_1 + \beta_2 = -0.213\)). Practices reduce
NPLs throughout (\(\beta_3 = -0.582\), \(p < 0.001\)). Economically, a
10-point CGDI increase post-reform cuts NPLs by 2.1 percentage points
(30\% of the sample mean).

Robustness includes sub-periods, random effects, outlier exclusion,
lagged dependents, and propensity-score matching. Mechanisms align with
signalling theory and institutional activation.

\textbf{Contributions:} (1) First Nigeria pre-/post-analysis; (2)
Practices dominate disclosure pre-reform; (3) Reforms as credible
exogenous shock, advancing natural-experiment methods in EM banking
\citep{pathan2009}. \textbf{Policy:} EM regulators prioritize enforceable
practices over voluntary disclosure \citep{bebchuk2010}.

% ==================== HYPOTHESES ====================
\subsection*{Hypotheses}

Building on agency theory \citep{jensen1976}, signalling theory
\citep{bushman2003}, and institutional voids literature
\citep{laporta1998,klapper2004}, we formalize two testable hypotheses:

\textbf{Hypothesis 1 (H$_1$): Institutional Activation Hypothesis} \\
The relationship between corporate governance disclosure (CGDI) and
non-performing loan ratios (NPLR) is statistically insignificant in the
pre-reform period (2009--2014) but becomes significantly negative
post-reform (2016--2024), due to the exogenous enhancement of disclosure
credibility via CBN-mandated enforcement and compliance verification.

\textbf{Rationale:} In weak institutional environments, voluntary
disclosure is prone to ``cheap talk'' \citep{klapper2004}. The CBN
reforms—through mandatory board term limits, independent director
quotas, and fit-and-proper vetting—transformed disclosure from symbolic
to substantive by increasing enforcement and verifiability, thereby
enabling market discipline and reducing moral hazard in credit
allocation.

\textbf{Hypothesis 2 (H$_2$): Practice Dominance Hypothesis} \\
Corporate governance practices (PIND), as measured by structural board
reforms and executive accountability mechanisms, exert a consistently
negative and statistically significant effect on NPL ratios across both
pre- and post-reform periods, with a stronger marginal impact than
disclosure in the pre-reform era.

\textbf{Rationale:} Unlike disclosure, actual governance practices
(e.g., board independence, tenure caps) directly constrain managerial
opportunism and enhance credit risk oversight, even absent strong
external enforcement \citep{pathan2009,bebchuk2010}.

\chapter{Main Results}

\begin{table}[htbp]
\centering
\caption{Fixed-Effects Panel Regression}
\label{tab:main}
\begin{tabular}{@{}lcc@{}}
\toprule
 & (1) & (2) \\ \midrule
CGDI                & 0.046      & 0.042      \\
                    & (0.074)    & (0.072)    \\
CGDI $\times$ Post2015 & $-0.259^{**}$ & $-0.261^{**}$ \\
                    & (0.099)    & (0.098)    \\
PIND                & $-0.582^{***}$ &            \\
                    & (0.149)    &            \\
COMID               &            & 0.175      \\
                    &            & (0.154)    \\
Market Cap          & $0.003^{**}$ & $0.003^{**}$ \\
                    & (0.001)    & (0.001)    \\
TCAR                & $-0.232^{**}$ & $-0.235^{**}$ \\
                    & (0.093)    & (0.092)    \\ \midrule
Observations        & 160        & 160        \\
$R^2$ (within)      & 0.341      & 0.339      \\
Bank FE             & Yes        & Yes        \\
Year FE             & Yes        & Yes        \\ \bottomrule
\end{tabular}
\begin{tablenotes}\small
\item Standard errors (clustered by bank) in parentheses.
\item $^{***}p<0.01$, $^{**}p<0.05$, $^{*}p<0.1$.
\end{tablenotes}
\end{table}

Column (1) includes all governance indices. The pre-reform CGDI
coefficient is insignificant ($\beta_1 = 0.046$, $p = 0.535$),
confirming ``box-ticking.'' The interaction term is negative and
significant ($\beta_2 = -0.259$, $p = 0.010$), yielding a total
post-reform effect of $-0.213$. A 10-point CGDI increase post-reform
reduces NPLR by 2.1 percentage points (30\% of the sample mean of 7.0\%).
The Practice Index dominates ($\beta = -0.582$, $p < 0.001$).

% ==================== HYPOTHESIS TESTING ====================
\subsection{Hypothesis Testing and Interpretation}

We now evaluate the two hypotheses using the difference-in-differences
framework in Equation~\eqref{eq:baseline}.

\subsubsection*{H$_1$: Institutional Activation – Supported}
The pre-reform coefficient on CGDI is positive but insignificant
($\beta_1 = 0.046$, $p = 0.535$), consistent with disclosure being
non-informative or ``box-ticking'' in a weak institutional setting. The
interaction term CGDI $\times$ Post2015 is negative and highly
significant ($\beta_2 = -0.259$, $p = 0.010$). The total post-reform
effect is:
\[
\beta_1 + \beta_2 = 0.046 - 0.259 = -0.213 \quad (p < 0.01).
\]
A 10-point increase in CGDI post-reform reduces NPLR by
\textbf{2.13 percentage points} (30\% of pre-reform mean), confirming
that the CBN reforms \textit{activated} disclosure as a credible signal.
Figure~\ref{fig:scatter} visually corroborates this structural break.

\subsubsection*{H$_2$: Practice Dominance – Supported}
The coefficient on PIND is large, negative, and significant throughout
($\beta_3 = -0.582$, $p < 0.001$). Pre-reform, practices reduce NPLR by
5.82 percentage points per 10-point increase—\textbf{more than double}
the post-reform disclosure effect. This dominance persists post-reform,
underscoring that \textit{enforceable board structure} matters more than
transparency alone in emerging markets.

\begin{figure}[htbp]
\centering
\includegraphics[width=0.95\textwidth]{npl_vs_cgdi_final.png}
\caption{NPL Ratio vs Corporate Governance Disclosure Index (CGDI):
Pre-2015 (blue, flat) vs Post-2015 (red, steep negative). The graph
shows a clear structural break in the relationship after the 2014--2016
CBN reforms, providing visual evidence for \textbf{H$_1$
(Institutional Activation)}. Before 2015, higher CGDI had no impact on
NPLs (near-flat trend). After 2015, the same increase in CGDI is
associated with a sharp decline in NPLs, confirming that reforms
transformed disclosure from ``cheap talk'' to a credible governance
signal.}
\label{fig:scatter}
\end{figure}

Figure~\ref{fig:scatter} visualises the structural break in the
relationship between transparency (CGDI) and credit risk (NPLR).
Pre-reform (blue points and trendline), the slope is nearly flat,
indicating that higher disclosure scores did not reduce NPLs—consistent
with ``cheap talk'' in weak institutional environments
\citep{klapper2004}. Post-reform (red points and trendline), the slope
turns sharply negative, showing that a one-standard-deviation increase
in CGDI (12.1 points) now lowers NPLR by 2.58 percentage points—a 36\%
reduction relative to the pre-reform mean. This visual evidence
complements the regression results and supports the identification
strategy: the CBN reforms exogenously shifted the information value of
disclosure.

\chapter{References}
\bibliography{refs}

\end{document}
